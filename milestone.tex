\title{Identifying Breast Cancer Gene Signatures with Feature Selection}
\author{
        \textsc{Gustavo Empinotti}
            \qquad
        \textsc{Susan Tu}
            \qquad
        \textsc{Raf Mertens}
        \mbox{}\\ %
        \\
        CS229\\
        \mbox{}\\ %
        \normalsize
            \texttt{gustavoe}
        \textbar{}
            \texttt{susanctu}
        \textbar{}
            \texttt{rafm}
        \normalsize
            \texttt{@stanford.edu}
}
\date{\today}
\documentclass[11pt]{article}
%\documentclass{acmconf}

\usepackage[paper=a4paper,dvips,top=1.5cm,left=1.5cm,right=1.5cm,
    foot=1cm,bottom=1.5cm]{geometry}
%----------------------------------------------------------
\begin{document}

\maketitle

\begin{abstract}
Statistical or machine learning methods have been used to find cancer gene signatures intended to predict survival or to classify patient prognosis as good/poor. However, a recent paper suggests that random gene subsets perform as well or better than 47 published gene signatures obtained through statistical methods, perhaps because cancer disrupts expression of many genes not directly related to cancer \cite{venet}. In this paper, we use different feature selection methods in an attempt to find a small set of genes that distinguishes normal breast tissue from malignant tumor tissue. We plan to compare the predictive power of our selected genes to that of random gene subsets, gene subsets that should be unrelated to cancer, and previously published gene signatures.
\end{abstract}

\section{Introduction}

\section{Methods}


\section{Discussion}

\paragraph{Forward Selection}
\paragraph{Backwards Selection}
\paragraph{Logistic Regression}

\begin{thebibliography}{9}

\bibitem{venet}
  David Venet, Jacques E. Dumont, Vincent Detours
  Most Random Gene Expression Signatures are Significantly Associated with Breast Cancer Outcome
\end{thebibliography}
\end{document}
